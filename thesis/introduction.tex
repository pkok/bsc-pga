\section{Introduction}
\label{ch:introduction}

\comment{Needs to be written. Will be done in the end.}

\TODO{Cite \cite{Hongbo}, \cite[Chapter 2 and 3]{Pottmann}, \cite{Barrau1, Barrau2}}

%Just as linear algebra, geometric algebra is an algebra over a given vector space $\reals^n$.  The difference lies in its operations; whereas linear algebra relies on matrix manipulations which cannot always be inverted, geometric algebra's base operation is the geometric product.  From this invertible product, one can deduce the inner product, known from linear algebra, and the outer product.  The outer product grants the user access to the exterior algebra $\bigwedge \reals^n$.  The outer product of $\V{A}, \V{B} \in \bigwedge \reals^n: \V{A} \wedge \V{B}$ represents the set of elements that are linear combinations of its operands $\left\{\alpha \V{A} + \beta \V{B} \mid \alpha, \beta \in \reals \right\}$
%
%There are already several models of geometry in use in the geometric algebra community, of which the conformal model is the most popular and widely known~\cite{TheBook}.  Although usable in many cases, the transformations of the conformal model are a subset of those of the projective model. The projective transformations are an important class of operations within computer vision and computer graphics. 
%
%GAViewer is a visualization and computing tool for the 3-dimensional Euclidean and conformal models of geometric algebra, developed by Daniel Fontijne at the University of Amsterdam~\cite{GAViewer}.  It allows the user to perform calculations in geometric algebra, and shows the results, both numerical as well as graphical.  Moreover, it allows the user to rotate, translate and zoom the viewport, as well as to manipulate the displayed elements.  The variables in which they are stored are automatically updated, and, if desired by the user, other objects that are parameterized by the manipulated object, may be updated dynamically as well.
%
%Recently, Li and Zhang~\cite{Hongbo} have found a way to model projective geometry, using a 6 dimensional representation space $\reals^{3,3}$ with a special metric, which allows three basis vectors to square to $-1$.  Using Pl\"ucker coordinates~(\cite{Hongbo},\cite[Chapter 2]{Pottmann}), lines in 3-dimensional Euclidean space $\mathbb{E}^3$ are represented in the representation space by vectors $\V{v}^2 = \V{v} \V{v} = \V{v} \cdot \V{v} = 0$.  As a consequence, there are also vectors in the representation space $\V{w}^2 \not= 0$ which do not represent lines in $\mathbb{E}^3$.  Different geometrical objects are generated by the outer product over lines and non-lines.  For example, for two intersecting lines $\ell_1, \ell_2$, the outer product $\ell_1 \wedge \ell_2$ represents a pencil, the collection of all lines that are in the same plane as $\ell_1$ and $\ell_2$, and pass through the same point.  Taking the outer product of this pencil with a third intersecting line that does not lie in the same plane, results in a bundle; the set of all lines intersecting in a certain point.  This could be used to represent points in our space of lines.
%
%In their article, Li and Zhang have not discussed what each element in their algebra might represent.  Barrau~\cite{Barrau1,Barrau2} and Pottmann and Wallner~\cite[Chapter 2 and 3]{Pottmann} have investigated the objects that can be represented in a 6-dimensional space with Pl\"ucker coordinates, both using a different algebra from ours.  Pottmann and Wallner use linear algebra, while Barrau's algebra has less expressive power.  


\subsection{Project aim and document structure}
\TODO{Needs work.  Don't talk about project aim, but research question.}

\comment{I quote this from the first line of the introduction.  Put ``quotes'' around?} 
GAViewer is a multi-purpose program for performing geometric algebra computations and visualizing geometric algebra~\cite{GAViewer}.  The aim of this project is to augment GAViewer so it can interpret the Pl\"ucker model for geometric algebra.  Besides knowing how to evaluate textual expressions to objects of the algebra, GAViewer needs new interpretation and visualization algorithms.

In \autoref{ch:background}, a short introduction to geometric algebra and the Pl\"ucker model for linear algebra are given.  The algebra and model are coupled by interpreting the inner product with the help of the homogeneous model.  \Autoref{ch:research} demonstrates the geometric interpretation of several blades of the Pl\"ucker model.  The computation of characteristics needed to draw certain blades together with their sensible visualization are described in \autoref{ch:implementation}.

This document concludes at \autoref{ch:conclusion} with a discussion of this work as well as possible future work, and a bibliography.
