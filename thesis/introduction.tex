\section{Introduction}
\label{ch:introduction}
Just as linear algebra, geometric algebra is an algebra over a given vector space $\reals^n$.  The difference lies in its operations; whereas linear algebra relies on matrix manipulations, geometric algebra's base operation is the geometric product.  From this product, one can deduce the inner product, known from linear algebra, and the outer product.  The outer product of two vectors $\V{a}$ and $\V{b}$ represents the set of elements that are linear combinations of its operands $\left\{\alpha \V{a} + \beta \V{b} \mid \alpha, \beta \in \reals \right\}$.

There are several models of geometry in use in the geometric algebra community, of which the conformal model is the most popular and widely known~\cite{TheBook}.  Although its angle-preserving transformations are usable in many cases, some problems need to be tackled by projective transformations.  These transformations are an important class of operations within computer vision and computer graphics. 

GAViewer is a visualization and computing tool for the 3-dimensional Euclidean, homogeneous and conformal models of geometric algebra, developed by Daniel Fontijne at the University of Amsterdam~\cite{GAViewer}.  It allows the user to perform calculations in selected models of geometric algebra, and shows the results, both numerically and graphically.  Moreover, it allows the user to rotate, translate and zoom the viewport, as well as to manipulate the displayed elements.  The variables in which they are stored are automatically updated, and, if desired by the user, other objects that are parameterized by the manipulated object may be updated dynamically as well.

Recently, Li and Zhang~\cite{Hongbo} have found a way to model projective geometry, using a 6 dimensional representation space $\reals^{3,3}$ with a special metric structure, which allows three of its basis vectors to square to $-1$.  Using Pl\"ucker coordinates~\cite{TheBook,Hongbo,Pottmann,Shoemake}, weighted lines are represented in the representation space by vectors $v$ satisfying the Pl\"ucker condition $\pluckerid(v) = v^2 = v v = v \dotp v = 0$.  As a consequence, there are also vectors $w$ in the representation space with $w^2 \not= 0$ which do not represent lines.  Objects with different geometrical interpretations are generated by the outer product over lines and non-lines.  For example, for two intersecting lines $a$ and $b$, the outer product represents a pencil of lines; the set of all lines that are in the same plane as $a$ and $b$, and pass through the same point.  The outer product of two skew lines $a'$ and $b'$ is interpreted as the set of those two lines.

In their article, Li and Zhang have not discussed what each element in their algebra might represent.  Barrau~\cite{Barrau1,Barrau2} and Pottmann and Wallner~\cite[Chapter 2 and 3]{Pottmann} have investigated the objects that can be represented in a 6-dimensional space with Pl\"ucker coordinates, both using different algebras from ours.  Pottmann and Wallner use linear algebra and Grassmann algebra, while Barrau's algebra has less expressive power.


\subsection{Project aim and document structure}
GAViewer is a multi-purpose program for performing geometric algebra computations and visualizing geometric algebra~\cite{GAViewer}.  The aim of this project is to augment GAViewer so it can interpret the Pl\"ucker model for geometric algebra.  Besides knowing how to evaluate textual expressions to objects of the algebra, GAViewer needs new interpretation and visualization algorithms.  For this, we show the relation between many blades of the model presented by Li and Zhang~\cite{Hongbo} and the elements described by Pottmann and Wallner~\cite{Pottmann}.

In \autoref{ch:background}, a brief introduction to geometric algebra and the Pl\"ucker model for linear algebra are given.  The algebra and model are coupled by interpreting the inner product with the help of the homogeneous model.  \Autoref{ch:research} demonstrates the geometric interpretation of several blades of the Pl\"ucker model.  The computation of characteristics needed to draw certain blades together with their sensible visualization are described in \autoref{ch:implementation}.  A rendering of a representative of each class of geometrically different objects is presented in \autoref{ch:visualization}.

\Autoref{ch:conclusion} presents the reader a summary of the results of this thesis, together with a discussion of this work as well as possible future work.  The document concludes with a bibliography.
