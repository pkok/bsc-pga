\section{Projective geometry and geometric algebra}
\label{ch:background}

\subsection{Geometric algebra}
\label{sec:intro-ga}
\comment{Paraphrase Moos Hueting's work?  Nah, I can do better.}

\comment{Include the next notation explanation:}
\begin{itemize}
  \item Bold for Euclidean
  \item Capital for multivectors
  \item Lowercase for vectors/1-blades
  \item Greek for scalars/0-blades
  \item $\reals^{3,3}$
  \item Geometric product: $A \gp B$ $\leftarrow$ small space
  \item Geometric division $A \gpi B$ or too trivial?
  \item Outer product $\wedge$
  \item Left contraction $\lcont$
  \item Dot product $\dotp$
  \item Dualization $\dual{A}$, undualization $\undual{A}$
  \item Euclidean (un-)dualization $\edual{\V{A}}$, $\eundual{\V{A}}$ or introduce extra notation $\V{A}^{\star}$, $\V{A}^{-\star}$?
  %\item Euclidean inproduct $\elcont$? $A \edotp B$, $A \elcont B$, $A \ercont B$\ldots
  \item \ldots
  \item Show that geometric product, cross product can be expressed in terms of $\wedge$ and $\lcont$ for $\Em$.
\end{itemize}

\subsection{Pl\"ucker model}
\label{sec:hongbo}
\label{sec:plucker}
\comment{This part will be paraphrased from \cite{TheBook} and \cite{Hongbo}.  Where do I put the ref to \cite{TheBook}?}

\comment{Are all weighted lines directed lines?}

Weighted infinitely-extending lines of $\reals^3$ have 5 degrees of freedom.  Such a line can be fully described by five numeric quantities. First, take a vector $\V{p}$ to a point on the line.  The point should be chosen so that the vector is orthogonal to the line.  The direction and weight of the line is given by a vector $\V{d}$.  These vectors are denoted in \autoref{fig:linedef}.

\begin{figure}
  \caption{A weighted line $L$ can be defined by any two of its weighted direction $\V{d}$, a vector $\V{p}$ orthogonal to the line, and its moment $\V{m} = \V{p} \times \V{d}$.}
  \label{fig:linedef}
  \begin{center}
    \fbox{
      \includegraphics[width=0.6\textwidth]{linedef}
    }
  \end{center}
\end{figure}

In the homogeneous model, the line through points $p = \ez + \V{p}$ and $q = \ez + \V{q}$ is represented as 
\begin{equation*}
  L = p \wedge q = \ez \wedge (\V{q} - \V{p}) + \V{p} \wedge \V{q} ,
\end{equation*}
with the following dependency relationship: 
\begin{equation} \label{eq:gaplucker0} 
  \ez \wedge (\V{q} - \V{p}) \wedge (\V{p} \wedge \V{q}) = 0 .
\end{equation}

This relationship limits the number of degrees of freedom from 6 to 5.

The direction and moment of a line are easily recognised in this expression.  The direction $\V{d} = \V{p} - \V{q}$ is encoded in the first factor as $\ez \wedge -\V{d} = \V{d} \wedge \ez = \V{d} \ez$.  
The moment $\V{m} = \V{p} \times \V{q} = \edual{(\V{p} \wedge \V{q})}$ can be found in the second term as $\eundual{\V{m}} = \eundual{(\edual{(\V{p} \wedge \V{q})})} = \V{p} \wedge \V{q}$, which results in another general formula for lines:
\begin{equation*}
  L = \V{d} \ez + \eundual{\V{m}}.
\end{equation*}

Within classical literature of linear algebra, the same object is written as $-\plucker{\V{d}}{\V{m}}$, using \emph{Pl\"ucker coordinates} to denote a line as a 6D vector.  The constraint of \autoref{eq:gaplucker0} is expressed as 
\begin{equation} \label{eq:laplucker0}
  \V{d} \dotp \V{m} = 0 .
\end{equation}

The set of 6D vectors $-\plucker{\V{d}}{\V{m}}$ that comply with this constraint correspond to the points on the Klein quadric.

The Pl\"ucker coordinates of a line are often treated as just six slots for storing numbers.  In most cases, operations on these elements are defined to manipulate two 3D vectors, \V{d} and \V{m}, instead of the whole 6D element.

Because of this, the user is often unaware of most of the algebraic structure.  In fact, this 6D vector corresponds to
\begin{equation*}
  L = d_1 \ez\ee + d_2 \ez\et + d_3 \ez\ed + m_1 \et\ed + m_2 \ed\ee + m_3 \ee\et .
\end{equation*}

The set $\{\ez\ee, \ez\et, \ez\ed, \et\ed, \ed\ee, \ee\et\}$ forms an orthogonal and complete basis of the bivectors of the homogeneous model of $\reals^3$.  It is orthogonal, because for any two elements $x, y$ it holds that $x \lcont y = 0$. It is also complete; it contains $(^4_2) = 6$ linear independent elements.

The Pl\"ucker model uses these six elements as its basis; it treats these bivectors as 1-dimensional elements.  These elements will be known as $\{\eze, \ezt, \ezd, \etd, \ede, \eet\}$.  Li and Zhang~\cite{Hongbo} define an embedding from the Pl\"ucker model to the homogeneous model:
\begin{equation*}
  \Em(x) = \left\{ 
    \begin{array}{ll}
      \ez \wedge \V{e}_i &\mbox{if $x = e_{0i}$}; \\
      \V{e}_i \wedge \V{e}_j &\mbox{if $x = e_{ij}$}; \\
      \Em(y) + \Em(z) &\mbox{if $x = y + z$}; \\
      \Em(y) \wedge \Em(z) &\mbox{if $x = y \wedge z$}; \\
      \left[\Em(y) \wedge \Em(z)\right] &\mbox{if $x = y \lcont z$}. \\
    \end{array}
    \right.
\end{equation*}

In the last line, the inner product for the Pl\"ucker is defined.  The bracket returns the proportionality factor of $\Em(y) \wedge \Em(z)$ to the homogeneous pseudoscalar $\ez \V{I}_3$.  This metric gives the multiplication table of \autoref{tab:nullmetric}.  Li and Zhang show that lines correspond with the null vectors of this space.  Because each of its basis elements correspond to a null vector, this basis is called the null basis. 

\begin{table}
  \caption{The multiplication table of the inner product for the Pl\"ucker model on the null basis.}
  \label{tab:nullmetric}
  \begin{center}
    \begin{tabular}{|c||c|c|c|c|c|c|}
      \hline
      $\lcont$ & $\eze$ & $\ezt$ & $\ezt$ & $\etd$ & $\ede$ & $\eet$ \\
      \hline \hline
      $\eze$ & 0 & 0 & 0 & 1 & 0 & 0 \\
      \hline
      $\ezt$ & 0 & 0 & 0 & 0 & 1 & 0 \\
      \hline
      $\ezd$ & 0 & 0 & 0 & 0 & 0 & 1 \\
      \hline
      $\etd$ & 1 & 0 & 0 & 0 & 0 & 0 \\
      \hline
      $\ede$ & 0 & 1 & 0 & 0 & 0 & 0 \\
      \hline
      $\eet$ & 0 & 0 & 1 & 0 & 0 & 0 \\
      \hline
    \end{tabular}
  \end{center}
\end{table}

Li and Zhang also show that the 6D space has the metric structure of $\reals^{3,3}$.  To demonstrate this structure, a second basis is given:

\begin{equation*}
  \begin{split}
  \left\{\ap, \bp, \cp, \am, \bm, \cm\right\} =
    & \left\{ \frac{\eze + \etd}{\sqrt{2}}, \frac{\ezt + \ede}{\sqrt{2}}, \frac{\ezd + \eet}{\sqrt{2}}, \right.\\
    & \left.  \frac{\eze - \etd}{\sqrt{2}}, \frac{\ezt - \ede}{\sqrt{2}}, \frac{\ezd - \eet}{\sqrt{2}}, \right\}
.
\end{split}
\end{equation*}

Without changing the semantics of the inner product, we obtain the multiplication table of \autoref{tab:pmmetric}.  It is apparent that the metric structure is $\reals^{3,3}$; three basis vectors, $\ap, \bp, \cp$, square to $1$, while the other three basis vectors, $\am, \bm, \cm$ square to $-1$.  

It is also clear that the basis elements do not represent lines, as no basis vectors squares to $0$.  \Autoref{ch:research} demonstrates that all vectors of the Pl\"ucker model represent screw motions.  Therefore, we name this basis the screw basis.

\begin{table}
  \caption{The multiplication table of the inner product for the Pl\"ucker model on the screw basis.}
  \label{tab:pmmetric}
  \begin{center}
    \begin{tabular}{|c||c|c|c|c|c|c|}
      \hline
      $\lcont$ & $\ap$ & $\bp$ & $\cp$ & $\am$ & $\bm$ & $\cm$ \\
      \hline \hline
      $\ap$ & 1 & 0 & 0 & 0 & 0 & 0 \\
      \hline
      $\bp$ & 0 & 1 & 0 & 0 & 0 & 0 \\
      \hline
      $\cp$ & 0 & 0 & 1 & 0 & 0 & 0 \\
      \hline
      $\am$ & 0 & 0 & 0 & -1 & 0 & 0 \\
      \hline
      $\bm$ & 0 & 0 & 0 & 0 & -1 & 0 \\
      \hline
      $\cm$ & 0 & 0 & 0 & 0 & 0 & -1 \\
      \hline
    \end{tabular}
  \end{center}
\end{table}
