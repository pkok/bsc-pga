\section{Classifying and parameterizing elements of different geometric interpretations}
\label{ch:research}

\comment{Write short introduction}

\subsection{The problem}
\label{sec:problem}
\comment{Needs a better subsection title.}

\subsection{The solution}
\label{sec:solution}
\comment{Needs a better subsection title.}

\subsection{Elements of grade 0 and 6}
Scalars are interpreted the same as in other models for geometric algebra.  They can be used to represent angles, weights, and many other scalar quantities.  Its dual, the pseudoscalar, has no special interpretation in this model as well.

\TODO{Explain a bit more about pseudoscalar? Scrap this section?}

\subsection{Elements of grade 1 and 5}
It is clear from \autoref{sec:plucker} that the null vectors of the Pl\"ucker model (those vectors $v$ which satisfy $v \lcont v = 0$) are interpreted as lines.  Let $\ell = \alpha_1 \eze + \alpha_2 \ezt + \alpha_3 \ezd + \beta_1 \etd + \beta_2 \ede + \beta_3 \eet$ be a null vector.  The line can be interpreted through the homogeneous model.  The direction of the line corresponds to $\Em(\alpha_1 \eze + \alpha_2 \ezt + \alpha_3 \ezd) \lcont -\ez = (\alpha_1 \ee + \alpha_2 \et + \alpha_3 \ed)$, while its moment is $\V{m} = \edual{\Em(\beta_1 \etd + \beta_2 \ede + \beta_3 \eet)} = \beta_1 \ee + \beta_2 \et + \beta_3 \ed$.

This only works with a nonzero direction.  The homogeneous model treats these Euclidean elements as bivectors.  In the Pl\"ucker model, there is no such thing as a purely Euclidean element, which leaves room for interpretation in a projective way.  These elements represent lines at infinity, or ideal lines.  An ideal line could be visualized as a circle which surrounds the space $\reals^3$ in the direction of $\Em(\ell) = \eundual{\V{m}}$.

\TODO{Connect interpretation with grade 5 objects}

\TODO{Interpret screws and interpret their duals}

\subsection{Elements of grade 2 and 4}

\subsection{Elements of grade 3}
